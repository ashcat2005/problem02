\documentclass[11pt]{article}
\pagestyle{plain}
\usepackage{latexsym,exscale,amsfonts,amsmath,amssymb,array}
\usepackage{color}
\usepackage[colorlinks]{hyperref}
\setlength{\topmargin}{-2.3cm}
\setlength{\textheight}{23.8cm}
\setlength{\oddsidemargin}{-0.5cm}
\setlength{\textwidth}{17cm}
\setlength{\parindent}{0cm}
\setlength{\parskip}{.4cm}
\newcommand{\totaldiffx}{\frac{d}{dx}}
\newcommand{\pardiffx}{\frac{\partial}{\partial x}}
\newcommand{\luft}{\:\!}

\usepackage{graphicx}
%\usepackage[latin1]{inputenc}
\usepackage{mathpazo}
\usepackage[T1]{fontenc}
\usepackage[comma,numbers,sort&compress]{natbib}
\usepackage[utf8]{inputenc}
\usepackage[spanish]{babel}


\begin{document}
\begin{center}
\large \bf Astrofísica Computacional\rm \\
2020\\
{\small Problema 2. Potencial Gravitacional en el interior de una Enana Blanca}
\end{center}

 {\bf Potencial Gravitacional en el interior de una Enana Blanca} \\
 
Dentro de los resultados del Problema 1, se encontró la densidad de masa $\rho(r)$ en el interior de una enana blanca. Con esta información, es posible encontrar el potencial gravitacional en el interior de este objeto astrofísico al solucionar la ecuación de Poisson,
\begin{equation}
\nabla^2 \phi (t, \textbf{r})= 4 \pi \rho(t, \textbf{r}).
\end{equation}

Aunque esta relación es una ecuación diferencial parcial, al imponer la simetría esférica que posee el sistema que se desea describir, se reduce a la ecuación radial
\begin{equation}
 \frac{1}{r^2} \frac{d}{dr} \left( r^2 \dfrac{d\phi (r)}{dr} \right) = 4 \pi \rho (r),
 \end{equation} 
en donde las derivadas parciales se han convertido en derivadas totales debido a la dependencia de las funciones. El problema diferencial queda completamente definido al imponer los valores de frontera,
\begin{align}
 \phi (0) = &0 \\
 \phi ( R) = & -\frac{GM}{R},
 \end{align} 
donde $M$ y $R$ son la masa y el radio de la enana blanca calculados en el Problema 1. 

\begin{enumerate}
\item Resuelva el problema de valores de frontera mediante el método shooting, utilizando un adecuado estimado inicial para la primera derivada y un método Runge-Kutta de orden 4.

\item Resuelva el problema de valores de frontera utilizando el método de diferencias-finitas.

\item Identifique si existe alguna diferencia en el potencial obtenido con los dos métodos. Grafique las soluciones obtenidas.
\end{enumerate}

\end{document}
